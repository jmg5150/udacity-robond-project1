
% Default to the notebook output style

    


% Inherit from the specified cell style.




    
\documentclass[11pt]{article}

    
    
    \usepackage[T1]{fontenc}
    % Nicer default font (+ math font) than Computer Modern for most use cases
    \usepackage{mathpazo}

    % Basic figure setup, for now with no caption control since it's done
    % automatically by Pandoc (which extracts ![](path) syntax from Markdown).
    \usepackage{graphicx}
    % We will generate all images so they have a width \maxwidth. This means
    % that they will get their normal width if they fit onto the page, but
    % are scaled down if they would overflow the margins.
    \makeatletter
    \def\maxwidth{\ifdim\Gin@nat@width>\linewidth\linewidth
    \else\Gin@nat@width\fi}
    \makeatother
    \let\Oldincludegraphics\includegraphics
    % Set max figure width to be 80% of text width, for now hardcoded.
    \renewcommand{\includegraphics}[1]{\Oldincludegraphics[width=.8\maxwidth]{#1}}
    % Ensure that by default, figures have no caption (until we provide a
    % proper Figure object with a Caption API and a way to capture that
    % in the conversion process - todo).
    \usepackage{caption}
    \DeclareCaptionLabelFormat{nolabel}{}
    \captionsetup{labelformat=nolabel}

    \usepackage{adjustbox} % Used to constrain images to a maximum size 
    \usepackage{xcolor} % Allow colors to be defined
    \usepackage{enumerate} % Needed for markdown enumerations to work
    \usepackage{geometry} % Used to adjust the document margins
    \usepackage{amsmath} % Equations
    \usepackage{amssymb} % Equations
    \usepackage{textcomp} % defines textquotesingle
    % Hack from http://tex.stackexchange.com/a/47451/13684:
    \AtBeginDocument{%
        \def\PYZsq{\textquotesingle}% Upright quotes in Pygmentized code
    }
    \usepackage{upquote} % Upright quotes for verbatim code
    \usepackage{eurosym} % defines \euro
    \usepackage[mathletters]{ucs} % Extended unicode (utf-8) support
    \usepackage[utf8x]{inputenc} % Allow utf-8 characters in the tex document
    \usepackage{fancyvrb} % verbatim replacement that allows latex
    \usepackage{grffile} % extends the file name processing of package graphics 
                         % to support a larger range 
    % The hyperref package gives us a pdf with properly built
    % internal navigation ('pdf bookmarks' for the table of contents,
    % internal cross-reference links, web links for URLs, etc.)
    \usepackage{hyperref}
    \usepackage{longtable} % longtable support required by pandoc >1.10
    \usepackage{booktabs}  % table support for pandoc > 1.12.2
    \usepackage[inline]{enumitem} % IRkernel/repr support (it uses the enumerate* environment)
    \usepackage[normalem]{ulem} % ulem is needed to support strikethroughs (\sout)
                                % normalem makes italics be italics, not underlines
    

    
    
    % Colors for the hyperref package
    \definecolor{urlcolor}{rgb}{0,.145,.698}
    \definecolor{linkcolor}{rgb}{.71,0.21,0.01}
    \definecolor{citecolor}{rgb}{.12,.54,.11}

    % ANSI colors
    \definecolor{ansi-black}{HTML}{3E424D}
    \definecolor{ansi-black-intense}{HTML}{282C36}
    \definecolor{ansi-red}{HTML}{E75C58}
    \definecolor{ansi-red-intense}{HTML}{B22B31}
    \definecolor{ansi-green}{HTML}{00A250}
    \definecolor{ansi-green-intense}{HTML}{007427}
    \definecolor{ansi-yellow}{HTML}{DDB62B}
    \definecolor{ansi-yellow-intense}{HTML}{B27D12}
    \definecolor{ansi-blue}{HTML}{208FFB}
    \definecolor{ansi-blue-intense}{HTML}{0065CA}
    \definecolor{ansi-magenta}{HTML}{D160C4}
    \definecolor{ansi-magenta-intense}{HTML}{A03196}
    \definecolor{ansi-cyan}{HTML}{60C6C8}
    \definecolor{ansi-cyan-intense}{HTML}{258F8F}
    \definecolor{ansi-white}{HTML}{C5C1B4}
    \definecolor{ansi-white-intense}{HTML}{A1A6B2}

    % commands and environments needed by pandoc snippets
    % extracted from the output of `pandoc -s`
    \providecommand{\tightlist}{%
      \setlength{\itemsep}{0pt}\setlength{\parskip}{0pt}}
    \DefineVerbatimEnvironment{Highlighting}{Verbatim}{commandchars=\\\{\}}
    % Add ',fontsize=\small' for more characters per line
    \newenvironment{Shaded}{}{}
    \newcommand{\KeywordTok}[1]{\textcolor[rgb]{0.00,0.44,0.13}{\textbf{{#1}}}}
    \newcommand{\DataTypeTok}[1]{\textcolor[rgb]{0.56,0.13,0.00}{{#1}}}
    \newcommand{\DecValTok}[1]{\textcolor[rgb]{0.25,0.63,0.44}{{#1}}}
    \newcommand{\BaseNTok}[1]{\textcolor[rgb]{0.25,0.63,0.44}{{#1}}}
    \newcommand{\FloatTok}[1]{\textcolor[rgb]{0.25,0.63,0.44}{{#1}}}
    \newcommand{\CharTok}[1]{\textcolor[rgb]{0.25,0.44,0.63}{{#1}}}
    \newcommand{\StringTok}[1]{\textcolor[rgb]{0.25,0.44,0.63}{{#1}}}
    \newcommand{\CommentTok}[1]{\textcolor[rgb]{0.38,0.63,0.69}{\textit{{#1}}}}
    \newcommand{\OtherTok}[1]{\textcolor[rgb]{0.00,0.44,0.13}{{#1}}}
    \newcommand{\AlertTok}[1]{\textcolor[rgb]{1.00,0.00,0.00}{\textbf{{#1}}}}
    \newcommand{\FunctionTok}[1]{\textcolor[rgb]{0.02,0.16,0.49}{{#1}}}
    \newcommand{\RegionMarkerTok}[1]{{#1}}
    \newcommand{\ErrorTok}[1]{\textcolor[rgb]{1.00,0.00,0.00}{\textbf{{#1}}}}
    \newcommand{\NormalTok}[1]{{#1}}
    
    % Additional commands for more recent versions of Pandoc
    \newcommand{\ConstantTok}[1]{\textcolor[rgb]{0.53,0.00,0.00}{{#1}}}
    \newcommand{\SpecialCharTok}[1]{\textcolor[rgb]{0.25,0.44,0.63}{{#1}}}
    \newcommand{\VerbatimStringTok}[1]{\textcolor[rgb]{0.25,0.44,0.63}{{#1}}}
    \newcommand{\SpecialStringTok}[1]{\textcolor[rgb]{0.73,0.40,0.53}{{#1}}}
    \newcommand{\ImportTok}[1]{{#1}}
    \newcommand{\DocumentationTok}[1]{\textcolor[rgb]{0.73,0.13,0.13}{\textit{{#1}}}}
    \newcommand{\AnnotationTok}[1]{\textcolor[rgb]{0.38,0.63,0.69}{\textbf{\textit{{#1}}}}}
    \newcommand{\CommentVarTok}[1]{\textcolor[rgb]{0.38,0.63,0.69}{\textbf{\textit{{#1}}}}}
    \newcommand{\VariableTok}[1]{\textcolor[rgb]{0.10,0.09,0.49}{{#1}}}
    \newcommand{\ControlFlowTok}[1]{\textcolor[rgb]{0.00,0.44,0.13}{\textbf{{#1}}}}
    \newcommand{\OperatorTok}[1]{\textcolor[rgb]{0.40,0.40,0.40}{{#1}}}
    \newcommand{\BuiltInTok}[1]{{#1}}
    \newcommand{\ExtensionTok}[1]{{#1}}
    \newcommand{\PreprocessorTok}[1]{\textcolor[rgb]{0.74,0.48,0.00}{{#1}}}
    \newcommand{\AttributeTok}[1]{\textcolor[rgb]{0.49,0.56,0.16}{{#1}}}
    \newcommand{\InformationTok}[1]{\textcolor[rgb]{0.38,0.63,0.69}{\textbf{\textit{{#1}}}}}
    \newcommand{\WarningTok}[1]{\textcolor[rgb]{0.38,0.63,0.69}{\textbf{\textit{{#1}}}}}
    
    
    % Define a nice break command that doesn't care if a line doesn't already
    % exist.
    \def\br{\hspace*{\fill} \\* }
    % Math Jax compatability definitions
    \def\gt{>}
    \def\lt{<}
    % Document parameters
    \title{Rover\_Project\_Test\_Notebook - Jonathan}
    
    
    

    % Pygments definitions
    
\makeatletter
\def\PY@reset{\let\PY@it=\relax \let\PY@bf=\relax%
    \let\PY@ul=\relax \let\PY@tc=\relax%
    \let\PY@bc=\relax \let\PY@ff=\relax}
\def\PY@tok#1{\csname PY@tok@#1\endcsname}
\def\PY@toks#1+{\ifx\relax#1\empty\else%
    \PY@tok{#1}\expandafter\PY@toks\fi}
\def\PY@do#1{\PY@bc{\PY@tc{\PY@ul{%
    \PY@it{\PY@bf{\PY@ff{#1}}}}}}}
\def\PY#1#2{\PY@reset\PY@toks#1+\relax+\PY@do{#2}}

\expandafter\def\csname PY@tok@ow\endcsname{\let\PY@bf=\textbf\def\PY@tc##1{\textcolor[rgb]{0.67,0.13,1.00}{##1}}}
\expandafter\def\csname PY@tok@nc\endcsname{\let\PY@bf=\textbf\def\PY@tc##1{\textcolor[rgb]{0.00,0.00,1.00}{##1}}}
\expandafter\def\csname PY@tok@sc\endcsname{\def\PY@tc##1{\textcolor[rgb]{0.73,0.13,0.13}{##1}}}
\expandafter\def\csname PY@tok@c\endcsname{\let\PY@it=\textit\def\PY@tc##1{\textcolor[rgb]{0.25,0.50,0.50}{##1}}}
\expandafter\def\csname PY@tok@s\endcsname{\def\PY@tc##1{\textcolor[rgb]{0.73,0.13,0.13}{##1}}}
\expandafter\def\csname PY@tok@sr\endcsname{\def\PY@tc##1{\textcolor[rgb]{0.73,0.40,0.53}{##1}}}
\expandafter\def\csname PY@tok@kd\endcsname{\let\PY@bf=\textbf\def\PY@tc##1{\textcolor[rgb]{0.00,0.50,0.00}{##1}}}
\expandafter\def\csname PY@tok@ch\endcsname{\let\PY@it=\textit\def\PY@tc##1{\textcolor[rgb]{0.25,0.50,0.50}{##1}}}
\expandafter\def\csname PY@tok@mo\endcsname{\def\PY@tc##1{\textcolor[rgb]{0.40,0.40,0.40}{##1}}}
\expandafter\def\csname PY@tok@cs\endcsname{\let\PY@it=\textit\def\PY@tc##1{\textcolor[rgb]{0.25,0.50,0.50}{##1}}}
\expandafter\def\csname PY@tok@mb\endcsname{\def\PY@tc##1{\textcolor[rgb]{0.40,0.40,0.40}{##1}}}
\expandafter\def\csname PY@tok@mf\endcsname{\def\PY@tc##1{\textcolor[rgb]{0.40,0.40,0.40}{##1}}}
\expandafter\def\csname PY@tok@nd\endcsname{\def\PY@tc##1{\textcolor[rgb]{0.67,0.13,1.00}{##1}}}
\expandafter\def\csname PY@tok@sh\endcsname{\def\PY@tc##1{\textcolor[rgb]{0.73,0.13,0.13}{##1}}}
\expandafter\def\csname PY@tok@w\endcsname{\def\PY@tc##1{\textcolor[rgb]{0.73,0.73,0.73}{##1}}}
\expandafter\def\csname PY@tok@gd\endcsname{\def\PY@tc##1{\textcolor[rgb]{0.63,0.00,0.00}{##1}}}
\expandafter\def\csname PY@tok@gu\endcsname{\let\PY@bf=\textbf\def\PY@tc##1{\textcolor[rgb]{0.50,0.00,0.50}{##1}}}
\expandafter\def\csname PY@tok@nn\endcsname{\let\PY@bf=\textbf\def\PY@tc##1{\textcolor[rgb]{0.00,0.00,1.00}{##1}}}
\expandafter\def\csname PY@tok@vc\endcsname{\def\PY@tc##1{\textcolor[rgb]{0.10,0.09,0.49}{##1}}}
\expandafter\def\csname PY@tok@mi\endcsname{\def\PY@tc##1{\textcolor[rgb]{0.40,0.40,0.40}{##1}}}
\expandafter\def\csname PY@tok@m\endcsname{\def\PY@tc##1{\textcolor[rgb]{0.40,0.40,0.40}{##1}}}
\expandafter\def\csname PY@tok@sb\endcsname{\def\PY@tc##1{\textcolor[rgb]{0.73,0.13,0.13}{##1}}}
\expandafter\def\csname PY@tok@kn\endcsname{\let\PY@bf=\textbf\def\PY@tc##1{\textcolor[rgb]{0.00,0.50,0.00}{##1}}}
\expandafter\def\csname PY@tok@s2\endcsname{\def\PY@tc##1{\textcolor[rgb]{0.73,0.13,0.13}{##1}}}
\expandafter\def\csname PY@tok@s1\endcsname{\def\PY@tc##1{\textcolor[rgb]{0.73,0.13,0.13}{##1}}}
\expandafter\def\csname PY@tok@sd\endcsname{\let\PY@it=\textit\def\PY@tc##1{\textcolor[rgb]{0.73,0.13,0.13}{##1}}}
\expandafter\def\csname PY@tok@kp\endcsname{\def\PY@tc##1{\textcolor[rgb]{0.00,0.50,0.00}{##1}}}
\expandafter\def\csname PY@tok@bp\endcsname{\def\PY@tc##1{\textcolor[rgb]{0.00,0.50,0.00}{##1}}}
\expandafter\def\csname PY@tok@err\endcsname{\def\PY@bc##1{\setlength{\fboxsep}{0pt}\fcolorbox[rgb]{1.00,0.00,0.00}{1,1,1}{\strut ##1}}}
\expandafter\def\csname PY@tok@c1\endcsname{\let\PY@it=\textit\def\PY@tc##1{\textcolor[rgb]{0.25,0.50,0.50}{##1}}}
\expandafter\def\csname PY@tok@sx\endcsname{\def\PY@tc##1{\textcolor[rgb]{0.00,0.50,0.00}{##1}}}
\expandafter\def\csname PY@tok@gi\endcsname{\def\PY@tc##1{\textcolor[rgb]{0.00,0.63,0.00}{##1}}}
\expandafter\def\csname PY@tok@vg\endcsname{\def\PY@tc##1{\textcolor[rgb]{0.10,0.09,0.49}{##1}}}
\expandafter\def\csname PY@tok@no\endcsname{\def\PY@tc##1{\textcolor[rgb]{0.53,0.00,0.00}{##1}}}
\expandafter\def\csname PY@tok@dl\endcsname{\def\PY@tc##1{\textcolor[rgb]{0.73,0.13,0.13}{##1}}}
\expandafter\def\csname PY@tok@mh\endcsname{\def\PY@tc##1{\textcolor[rgb]{0.40,0.40,0.40}{##1}}}
\expandafter\def\csname PY@tok@fm\endcsname{\def\PY@tc##1{\textcolor[rgb]{0.00,0.00,1.00}{##1}}}
\expandafter\def\csname PY@tok@cp\endcsname{\def\PY@tc##1{\textcolor[rgb]{0.74,0.48,0.00}{##1}}}
\expandafter\def\csname PY@tok@go\endcsname{\def\PY@tc##1{\textcolor[rgb]{0.53,0.53,0.53}{##1}}}
\expandafter\def\csname PY@tok@ge\endcsname{\let\PY@it=\textit}
\expandafter\def\csname PY@tok@o\endcsname{\def\PY@tc##1{\textcolor[rgb]{0.40,0.40,0.40}{##1}}}
\expandafter\def\csname PY@tok@gr\endcsname{\def\PY@tc##1{\textcolor[rgb]{1.00,0.00,0.00}{##1}}}
\expandafter\def\csname PY@tok@k\endcsname{\let\PY@bf=\textbf\def\PY@tc##1{\textcolor[rgb]{0.00,0.50,0.00}{##1}}}
\expandafter\def\csname PY@tok@nt\endcsname{\let\PY@bf=\textbf\def\PY@tc##1{\textcolor[rgb]{0.00,0.50,0.00}{##1}}}
\expandafter\def\csname PY@tok@kr\endcsname{\let\PY@bf=\textbf\def\PY@tc##1{\textcolor[rgb]{0.00,0.50,0.00}{##1}}}
\expandafter\def\csname PY@tok@nl\endcsname{\def\PY@tc##1{\textcolor[rgb]{0.63,0.63,0.00}{##1}}}
\expandafter\def\csname PY@tok@vi\endcsname{\def\PY@tc##1{\textcolor[rgb]{0.10,0.09,0.49}{##1}}}
\expandafter\def\csname PY@tok@gs\endcsname{\let\PY@bf=\textbf}
\expandafter\def\csname PY@tok@sa\endcsname{\def\PY@tc##1{\textcolor[rgb]{0.73,0.13,0.13}{##1}}}
\expandafter\def\csname PY@tok@il\endcsname{\def\PY@tc##1{\textcolor[rgb]{0.40,0.40,0.40}{##1}}}
\expandafter\def\csname PY@tok@gt\endcsname{\def\PY@tc##1{\textcolor[rgb]{0.00,0.27,0.87}{##1}}}
\expandafter\def\csname PY@tok@se\endcsname{\let\PY@bf=\textbf\def\PY@tc##1{\textcolor[rgb]{0.73,0.40,0.13}{##1}}}
\expandafter\def\csname PY@tok@kt\endcsname{\def\PY@tc##1{\textcolor[rgb]{0.69,0.00,0.25}{##1}}}
\expandafter\def\csname PY@tok@cpf\endcsname{\let\PY@it=\textit\def\PY@tc##1{\textcolor[rgb]{0.25,0.50,0.50}{##1}}}
\expandafter\def\csname PY@tok@gp\endcsname{\let\PY@bf=\textbf\def\PY@tc##1{\textcolor[rgb]{0.00,0.00,0.50}{##1}}}
\expandafter\def\csname PY@tok@cm\endcsname{\let\PY@it=\textit\def\PY@tc##1{\textcolor[rgb]{0.25,0.50,0.50}{##1}}}
\expandafter\def\csname PY@tok@nf\endcsname{\def\PY@tc##1{\textcolor[rgb]{0.00,0.00,1.00}{##1}}}
\expandafter\def\csname PY@tok@kc\endcsname{\let\PY@bf=\textbf\def\PY@tc##1{\textcolor[rgb]{0.00,0.50,0.00}{##1}}}
\expandafter\def\csname PY@tok@vm\endcsname{\def\PY@tc##1{\textcolor[rgb]{0.10,0.09,0.49}{##1}}}
\expandafter\def\csname PY@tok@nv\endcsname{\def\PY@tc##1{\textcolor[rgb]{0.10,0.09,0.49}{##1}}}
\expandafter\def\csname PY@tok@nb\endcsname{\def\PY@tc##1{\textcolor[rgb]{0.00,0.50,0.00}{##1}}}
\expandafter\def\csname PY@tok@gh\endcsname{\let\PY@bf=\textbf\def\PY@tc##1{\textcolor[rgb]{0.00,0.00,0.50}{##1}}}
\expandafter\def\csname PY@tok@ni\endcsname{\let\PY@bf=\textbf\def\PY@tc##1{\textcolor[rgb]{0.60,0.60,0.60}{##1}}}
\expandafter\def\csname PY@tok@ss\endcsname{\def\PY@tc##1{\textcolor[rgb]{0.10,0.09,0.49}{##1}}}
\expandafter\def\csname PY@tok@ne\endcsname{\let\PY@bf=\textbf\def\PY@tc##1{\textcolor[rgb]{0.82,0.25,0.23}{##1}}}
\expandafter\def\csname PY@tok@si\endcsname{\let\PY@bf=\textbf\def\PY@tc##1{\textcolor[rgb]{0.73,0.40,0.53}{##1}}}
\expandafter\def\csname PY@tok@na\endcsname{\def\PY@tc##1{\textcolor[rgb]{0.49,0.56,0.16}{##1}}}

\def\PYZbs{\char`\\}
\def\PYZus{\char`\_}
\def\PYZob{\char`\{}
\def\PYZcb{\char`\}}
\def\PYZca{\char`\^}
\def\PYZam{\char`\&}
\def\PYZlt{\char`\<}
\def\PYZgt{\char`\>}
\def\PYZsh{\char`\#}
\def\PYZpc{\char`\%}
\def\PYZdl{\char`\$}
\def\PYZhy{\char`\-}
\def\PYZsq{\char`\'}
\def\PYZdq{\char`\"}
\def\PYZti{\char`\~}
% for compatibility with earlier versions
\def\PYZat{@}
\def\PYZlb{[}
\def\PYZrb{]}
\makeatother


    % Exact colors from NB
    \definecolor{incolor}{rgb}{0.0, 0.0, 0.5}
    \definecolor{outcolor}{rgb}{0.545, 0.0, 0.0}



    
    % Prevent overflowing lines due to hard-to-break entities
    \sloppy 
    % Setup hyperref package
    \hypersetup{
      breaklinks=true,  % so long urls are correctly broken across lines
      colorlinks=true,
      urlcolor=urlcolor,
      linkcolor=linkcolor,
      citecolor=citecolor,
      }
    % Slightly bigger margins than the latex defaults
    
    \geometry{verbose,tmargin=1in,bmargin=1in,lmargin=1in,rmargin=1in}
    
    

    \begin{document}
    
    
    \maketitle
    
    

    
    \subsection{Rover Project Test
Notebook}\label{rover-project-test-notebook}

This notebook contains the functions from the lesson and provides the
scaffolding you need to test out your mapping methods. The steps you
need to complete in this notebook for the project are the following:

\begin{itemize}
\tightlist
\item
  First just run each of the cells in the notebook, examine the code and
  the results of each.
\item
  Run the simulator in "Training Mode" and record some data. Note: the
  simulator may crash if you try to record a large (longer than a few
  minutes) dataset, but you don't need a ton of data, just some example
  images to work with.\\
\item
  Change the data directory path (2 cells below) to be the directory
  where you saved data
\item
  Test out the functions provided on your data
\item
  Write new functions (or modify existing ones) to report and map out
  detections of obstacles and rock samples (yellow rocks)
\item
  Populate the \texttt{process\_image()} function with the appropriate
  steps/functions to go from a raw image to a worldmap.
\item
  Run the cell that calls \texttt{process\_image()} using
  \texttt{moviepy} functions to create video output
\item
  Once you have mapping working, move on to modifying
  \texttt{perception.py} and \texttt{decision.py} to allow your rover to
  navigate and map in autonomous mode!
\end{itemize}

\textbf{Note: If, at any point, you encounter frozen display windows or
other confounding issues, you can always start again with a clean slate
by going to the "Kernel" menu above and selecting "Restart \& Clear
Output".}

\textbf{Run the next cell to get code highlighting in the markdown
cells.}

    \begin{Verbatim}[commandchars=\\\{\}]
{\color{incolor}In [{\color{incolor}1}]:} \PY{o}{\PYZpc{}\PYZpc{}}\PY{k}{HTML}
        \PYZlt{}style\PYZgt{} code \PYZob{}background\PYZhy{}color : orange !important;\PYZcb{} \PYZlt{}/style\PYZgt{}
\end{Verbatim}


    
    \begin{verbatim}
<IPython.core.display.HTML object>
    \end{verbatim}

    
    \begin{Verbatim}[commandchars=\\\{\}]
{\color{incolor}In [{\color{incolor} }]:} \PY{o}{\PYZpc{}}\PY{k}{matplotlib} inline
        \PY{c+c1}{\PYZsh{}\PYZpc{}matplotlib qt \PYZsh{} Choose \PYZpc{}matplotlib qt to plot to an interactive window (note it may show up behind your browser)}
        \PY{c+c1}{\PYZsh{} Make some of the relevant imports}
        \PY{k+kn}{import} \PY{n+nn}{cv2} \PY{c+c1}{\PYZsh{} OpenCV for perspective transform}
        \PY{k+kn}{import} \PY{n+nn}{numpy} \PY{k}{as} \PY{n+nn}{np}
        \PY{k+kn}{import} \PY{n+nn}{matplotlib}\PY{n+nn}{.}\PY{n+nn}{image} \PY{k}{as} \PY{n+nn}{mpimg}
        \PY{k+kn}{import} \PY{n+nn}{matplotlib}\PY{n+nn}{.}\PY{n+nn}{pyplot} \PY{k}{as} \PY{n+nn}{plt}
        \PY{k+kn}{import} \PY{n+nn}{scipy}\PY{n+nn}{.}\PY{n+nn}{misc} \PY{c+c1}{\PYZsh{} For saving images as needed}
        \PY{k+kn}{import} \PY{n+nn}{glob}  \PY{c+c1}{\PYZsh{} For reading in a list of images from a folder}
        \PY{k+kn}{import} \PY{n+nn}{imageio}
        \PY{n}{imageio}\PY{o}{.}\PY{n}{plugins}\PY{o}{.}\PY{n}{ffmpeg}\PY{o}{.}\PY{n}{download}\PY{p}{(}\PY{p}{)}
\end{Verbatim}


    \subsection{Quick Look at the Data}\label{quick-look-at-the-data}

There's some example data provided in the \texttt{test\_dataset} folder.
This basic dataset is enough to get you up and running but if you want
to hone your methods more carefully you should record some data of your
own to sample various scenarios in the simulator.

Next, read in and display a random image from the \texttt{test\_dataset}
folder

    \begin{Verbatim}[commandchars=\\\{\}]
{\color{incolor}In [{\color{incolor}3}]:} \PY{n}{path} \PY{o}{=} \PY{l+s+s1}{\PYZsq{}}\PY{l+s+s1}{../test\PYZus{}dataset/IMG/*}\PY{l+s+s1}{\PYZsq{}}
        \PY{n}{img\PYZus{}list} \PY{o}{=} \PY{n}{glob}\PY{o}{.}\PY{n}{glob}\PY{p}{(}\PY{n}{path}\PY{p}{)}
        \PY{c+c1}{\PYZsh{} Grab a random image and display it}
        \PY{n}{idx} \PY{o}{=} \PY{n}{np}\PY{o}{.}\PY{n}{random}\PY{o}{.}\PY{n}{randint}\PY{p}{(}\PY{l+m+mi}{0}\PY{p}{,} \PY{n+nb}{len}\PY{p}{(}\PY{n}{img\PYZus{}list}\PY{p}{)}\PY{o}{\PYZhy{}}\PY{l+m+mi}{1}\PY{p}{)}
        \PY{n}{image} \PY{o}{=} \PY{n}{mpimg}\PY{o}{.}\PY{n}{imread}\PY{p}{(}\PY{n}{img\PYZus{}list}\PY{p}{[}\PY{n}{idx}\PY{p}{]}\PY{p}{)}
        \PY{n}{plt}\PY{o}{.}\PY{n}{imshow}\PY{p}{(}\PY{n}{image}\PY{p}{)}
\end{Verbatim}


\begin{Verbatim}[commandchars=\\\{\}]
{\color{outcolor}Out[{\color{outcolor}3}]:} <matplotlib.image.AxesImage at 0x1d28277eda0>
\end{Verbatim}
            
    \begin{center}
    \adjustimage{max size={0.9\linewidth}{0.9\paperheight}}{output_4_1.png}
    \end{center}
    { \hspace*{\fill} \\}
    
    \subsection{Calibration Data}\label{calibration-data}

Read in and display example grid and rock sample calibration images.
You'll use the grid for perspective transform and the rock image for
creating a new color selection that identifies these samples of
interest.

    \begin{Verbatim}[commandchars=\\\{\}]
{\color{incolor}In [{\color{incolor}4}]:} \PY{c+c1}{\PYZsh{} In the simulator you can toggle on a grid on the ground for calibration}
        \PY{c+c1}{\PYZsh{} You can also toggle on the rock samples with the 0 (zero) key.  }
        \PY{c+c1}{\PYZsh{} Here\PYZsq{}s an example of the grid and one of the rocks}
        \PY{n}{example\PYZus{}grid} \PY{o}{=} \PY{l+s+s1}{\PYZsq{}}\PY{l+s+s1}{../calibration\PYZus{}images/example\PYZus{}grid1.jpg}\PY{l+s+s1}{\PYZsq{}}
        \PY{n}{example\PYZus{}rock} \PY{o}{=} \PY{l+s+s1}{\PYZsq{}}\PY{l+s+s1}{../calibration\PYZus{}images/example\PYZus{}rock1.jpg}\PY{l+s+s1}{\PYZsq{}}
        \PY{n}{grid\PYZus{}img} \PY{o}{=} \PY{n}{mpimg}\PY{o}{.}\PY{n}{imread}\PY{p}{(}\PY{n}{example\PYZus{}grid}\PY{p}{)}
        \PY{n}{rock\PYZus{}img} \PY{o}{=} \PY{n}{mpimg}\PY{o}{.}\PY{n}{imread}\PY{p}{(}\PY{n}{example\PYZus{}rock}\PY{p}{)}
        
        \PY{n}{fig} \PY{o}{=} \PY{n}{plt}\PY{o}{.}\PY{n}{figure}\PY{p}{(}\PY{n}{figsize}\PY{o}{=}\PY{p}{(}\PY{l+m+mi}{12}\PY{p}{,}\PY{l+m+mi}{3}\PY{p}{)}\PY{p}{)}
        \PY{n}{plt}\PY{o}{.}\PY{n}{subplot}\PY{p}{(}\PY{l+m+mi}{121}\PY{p}{)}
        \PY{n}{plt}\PY{o}{.}\PY{n}{imshow}\PY{p}{(}\PY{n}{grid\PYZus{}img}\PY{p}{)}
        \PY{n}{plt}\PY{o}{.}\PY{n}{subplot}\PY{p}{(}\PY{l+m+mi}{122}\PY{p}{)}
        \PY{n}{plt}\PY{o}{.}\PY{n}{imshow}\PY{p}{(}\PY{n}{rock\PYZus{}img}\PY{p}{)}
\end{Verbatim}


\begin{Verbatim}[commandchars=\\\{\}]
{\color{outcolor}Out[{\color{outcolor}4}]:} <matplotlib.image.AxesImage at 0x1d282a2c2b0>
\end{Verbatim}
            
    \begin{center}
    \adjustimage{max size={0.9\linewidth}{0.9\paperheight}}{output_6_1.png}
    \end{center}
    { \hspace*{\fill} \\}
    
    \subsection{Perspective Transform}\label{perspective-transform}

Define the perspective transform function from the lesson and test it on
an image.

    \begin{Verbatim}[commandchars=\\\{\}]
{\color{incolor}In [{\color{incolor}5}]:} \PY{c+c1}{\PYZsh{} Define a function to perform a perspective transform}
        \PY{c+c1}{\PYZsh{} I\PYZsq{}ve used the example grid image above to choose source points for the}
        \PY{c+c1}{\PYZsh{} grid cell in front of the rover (each grid cell is 1 square meter in the sim)}
        \PY{c+c1}{\PYZsh{} Define a function to perform a perspective transform}
        \PY{k}{def} \PY{n+nf}{perspect\PYZus{}transform}\PY{p}{(}\PY{n}{img}\PY{p}{,} \PY{n}{src}\PY{p}{,} \PY{n}{dst}\PY{p}{)}\PY{p}{:}
                   
            \PY{n}{M} \PY{o}{=} \PY{n}{cv2}\PY{o}{.}\PY{n}{getPerspectiveTransform}\PY{p}{(}\PY{n}{src}\PY{p}{,} \PY{n}{dst}\PY{p}{)}
            \PY{n}{warped} \PY{o}{=} \PY{n}{cv2}\PY{o}{.}\PY{n}{warpPerspective}\PY{p}{(}\PY{n}{img}\PY{p}{,} \PY{n}{M}\PY{p}{,} \PY{p}{(}\PY{n}{img}\PY{o}{.}\PY{n}{shape}\PY{p}{[}\PY{l+m+mi}{1}\PY{p}{]}\PY{p}{,} \PY{n}{img}\PY{o}{.}\PY{n}{shape}\PY{p}{[}\PY{l+m+mi}{0}\PY{p}{]}\PY{p}{)}\PY{p}{)}\PY{c+c1}{\PYZsh{} keep same size as input image}
            
            \PY{k}{return} \PY{n}{warped}
        
        
        \PY{c+c1}{\PYZsh{} Define calibration box in source (actual) and destination (desired) coordinates}
        \PY{c+c1}{\PYZsh{} These source and destination points are defined to warp the image}
        \PY{c+c1}{\PYZsh{} to a grid where each 10x10 pixel square represents 1 square meter}
        \PY{c+c1}{\PYZsh{} The destination box will be 2*dst\PYZus{}size on each side}
        \PY{n}{dst\PYZus{}size} \PY{o}{=} \PY{l+m+mi}{5} 
        \PY{c+c1}{\PYZsh{} Set a bottom offset to account for the fact that the bottom of the image }
        \PY{c+c1}{\PYZsh{} is not the position of the rover but a bit in front of it}
        \PY{c+c1}{\PYZsh{} this is just a rough guess, feel free to change it!}
        \PY{n}{bottom\PYZus{}offset} \PY{o}{=} \PY{l+m+mi}{6}
        \PY{n}{source} \PY{o}{=} \PY{n}{np}\PY{o}{.}\PY{n}{float32}\PY{p}{(}\PY{p}{[}\PY{p}{[}\PY{l+m+mi}{14}\PY{p}{,} \PY{l+m+mi}{140}\PY{p}{]}\PY{p}{,} \PY{p}{[}\PY{l+m+mi}{301} \PY{p}{,}\PY{l+m+mi}{140}\PY{p}{]}\PY{p}{,}\PY{p}{[}\PY{l+m+mi}{200}\PY{p}{,} \PY{l+m+mi}{96}\PY{p}{]}\PY{p}{,} \PY{p}{[}\PY{l+m+mi}{118}\PY{p}{,} \PY{l+m+mi}{96}\PY{p}{]}\PY{p}{]}\PY{p}{)}
        \PY{n}{destination} \PY{o}{=} \PY{n}{np}\PY{o}{.}\PY{n}{float32}\PY{p}{(}\PY{p}{[}\PY{p}{[}\PY{n}{image}\PY{o}{.}\PY{n}{shape}\PY{p}{[}\PY{l+m+mi}{1}\PY{p}{]}\PY{o}{/}\PY{l+m+mi}{2} \PY{o}{\PYZhy{}} \PY{n}{dst\PYZus{}size}\PY{p}{,} \PY{n}{image}\PY{o}{.}\PY{n}{shape}\PY{p}{[}\PY{l+m+mi}{0}\PY{p}{]} \PY{o}{\PYZhy{}} \PY{n}{bottom\PYZus{}offset}\PY{p}{]}\PY{p}{,}
                          \PY{p}{[}\PY{n}{image}\PY{o}{.}\PY{n}{shape}\PY{p}{[}\PY{l+m+mi}{1}\PY{p}{]}\PY{o}{/}\PY{l+m+mi}{2} \PY{o}{+} \PY{n}{dst\PYZus{}size}\PY{p}{,} \PY{n}{image}\PY{o}{.}\PY{n}{shape}\PY{p}{[}\PY{l+m+mi}{0}\PY{p}{]} \PY{o}{\PYZhy{}} \PY{n}{bottom\PYZus{}offset}\PY{p}{]}\PY{p}{,}
                          \PY{p}{[}\PY{n}{image}\PY{o}{.}\PY{n}{shape}\PY{p}{[}\PY{l+m+mi}{1}\PY{p}{]}\PY{o}{/}\PY{l+m+mi}{2} \PY{o}{+} \PY{n}{dst\PYZus{}size}\PY{p}{,} \PY{n}{image}\PY{o}{.}\PY{n}{shape}\PY{p}{[}\PY{l+m+mi}{0}\PY{p}{]} \PY{o}{\PYZhy{}} \PY{l+m+mi}{2}\PY{o}{*}\PY{n}{dst\PYZus{}size} \PY{o}{\PYZhy{}} \PY{n}{bottom\PYZus{}offset}\PY{p}{]}\PY{p}{,} 
                          \PY{p}{[}\PY{n}{image}\PY{o}{.}\PY{n}{shape}\PY{p}{[}\PY{l+m+mi}{1}\PY{p}{]}\PY{o}{/}\PY{l+m+mi}{2} \PY{o}{\PYZhy{}} \PY{n}{dst\PYZus{}size}\PY{p}{,} \PY{n}{image}\PY{o}{.}\PY{n}{shape}\PY{p}{[}\PY{l+m+mi}{0}\PY{p}{]} \PY{o}{\PYZhy{}} \PY{l+m+mi}{2}\PY{o}{*}\PY{n}{dst\PYZus{}size} \PY{o}{\PYZhy{}} \PY{n}{bottom\PYZus{}offset}\PY{p}{]}\PY{p}{,}
                          \PY{p}{]}\PY{p}{)}
        \PY{n}{warped} \PY{o}{=} \PY{n}{perspect\PYZus{}transform}\PY{p}{(}\PY{n}{grid\PYZus{}img}\PY{p}{,} \PY{n}{source}\PY{p}{,} \PY{n}{destination}\PY{p}{)}
        \PY{n}{plt}\PY{o}{.}\PY{n}{imshow}\PY{p}{(}\PY{n}{warped}\PY{p}{)}
        \PY{c+c1}{\PYZsh{}scipy.misc.imsave(\PYZsq{}../output/warped\PYZus{}example.jpg\PYZsq{}, warped)}
\end{Verbatim}


\begin{Verbatim}[commandchars=\\\{\}]
{\color{outcolor}Out[{\color{outcolor}5}]:} <matplotlib.image.AxesImage at 0x1d282a85ba8>
\end{Verbatim}
            
    \begin{center}
    \adjustimage{max size={0.9\linewidth}{0.9\paperheight}}{output_8_1.png}
    \end{center}
    { \hspace*{\fill} \\}
    
    \subsection{Color Thresholding}\label{color-thresholding}

Define the color thresholding function from the lesson and apply it to
the warped image

\textbf{TODO:} Ultimately, you want your map to not just include
navigable terrain but also obstacles and the positions of the rock
samples you're searching for. Modify this function or write a new
function that returns the pixel locations of obstacles (areas below the
threshold) and rock samples (yellow rocks in calibration images), such
that you can map these areas into world coordinates as well.\\
\textbf{Hints and Suggestion:} * For obstacles you can just invert your
color selection that you used to detect ground pixels, i.e., if you've
decided that everything above the threshold is navigable terrain, then
everthing below the threshold must be an obstacle!

\begin{itemize}
\item
  For rocks, think about imposing a lower and upper boundary in your
  color selection to be more specific about choosing colors. You can
  investigate the colors of the rocks (the RGB pixel values) in an
  interactive matplotlib window to get a feel for the appropriate
  threshold range (keep in mind you may want different ranges for each
  of R, G and B!). Feel free to get creative and even bring in functions
  from other libraries. Here's an example of
  \href{http://opencv-python-tutroals.readthedocs.io/en/latest/py_tutorials/py_imgproc/py_colorspaces/py_colorspaces.html}{color
  selection} using OpenCV.
\item
  \textbf{Beware However:} if you start manipulating images with OpenCV,
  keep in mind that it defaults to \texttt{BGR} instead of \texttt{RGB}
  color space when reading/writing images, so things can get confusing.
\end{itemize}

    \begin{Verbatim}[commandchars=\\\{\}]
{\color{incolor}In [{\color{incolor}6}]:} \PY{c+c1}{\PYZsh{} Identify pixels above the threshold}
        \PY{c+c1}{\PYZsh{} Threshold of RGB \PYZgt{} 160 does a nice job of identifying ground pixels only}
        \PY{k}{def} \PY{n+nf}{color\PYZus{}thresh}\PY{p}{(}\PY{n}{img}\PY{p}{,} \PY{n}{rgb\PYZus{}thresh}\PY{o}{=}\PY{p}{(}\PY{l+m+mi}{160}\PY{p}{,} \PY{l+m+mi}{160}\PY{p}{,} \PY{l+m+mi}{160}\PY{p}{)}\PY{p}{)}\PY{p}{:}
            \PY{c+c1}{\PYZsh{} Create an array of zeros same xy size as img, but single channel}
            \PY{n}{color\PYZus{}select} \PY{o}{=} \PY{n}{np}\PY{o}{.}\PY{n}{zeros\PYZus{}like}\PY{p}{(}\PY{n}{img}\PY{p}{[}\PY{p}{:}\PY{p}{,}\PY{p}{:}\PY{p}{,}\PY{l+m+mi}{0}\PY{p}{]}\PY{p}{)}
            \PY{c+c1}{\PYZsh{} Require that each pixel be above all three threshold values in RGB}
            \PY{c+c1}{\PYZsh{} above\PYZus{}thresh will now contain a boolean array with \PYZdq{}True\PYZdq{}}
            \PY{c+c1}{\PYZsh{} where threshold was met}
            \PY{n}{above\PYZus{}thresh} \PY{o}{=} \PY{p}{(}\PY{n}{img}\PY{p}{[}\PY{p}{:}\PY{p}{,}\PY{p}{:}\PY{p}{,}\PY{l+m+mi}{0}\PY{p}{]} \PY{o}{\PYZgt{}} \PY{n}{rgb\PYZus{}thresh}\PY{p}{[}\PY{l+m+mi}{0}\PY{p}{]}\PY{p}{)} \PYZbs{}
                        \PY{o}{\PYZam{}} \PY{p}{(}\PY{n}{img}\PY{p}{[}\PY{p}{:}\PY{p}{,}\PY{p}{:}\PY{p}{,}\PY{l+m+mi}{1}\PY{p}{]} \PY{o}{\PYZgt{}} \PY{n}{rgb\PYZus{}thresh}\PY{p}{[}\PY{l+m+mi}{1}\PY{p}{]}\PY{p}{)} \PYZbs{}
                        \PY{o}{\PYZam{}} \PY{p}{(}\PY{n}{img}\PY{p}{[}\PY{p}{:}\PY{p}{,}\PY{p}{:}\PY{p}{,}\PY{l+m+mi}{2}\PY{p}{]} \PY{o}{\PYZgt{}} \PY{n}{rgb\PYZus{}thresh}\PY{p}{[}\PY{l+m+mi}{2}\PY{p}{]}\PY{p}{)}
            \PY{c+c1}{\PYZsh{} Index the array of zeros with the boolean array and set to 1}
            \PY{n}{color\PYZus{}select}\PY{p}{[}\PY{n}{above\PYZus{}thresh}\PY{p}{]} \PY{o}{=} \PY{l+m+mi}{1}
            \PY{c+c1}{\PYZsh{} Return the binary image}
            \PY{k}{return} \PY{n}{color\PYZus{}select}
        
        \PY{n}{threshed} \PY{o}{=} \PY{n}{color\PYZus{}thresh}\PY{p}{(}\PY{n}{warped}\PY{p}{)}
        \PY{n}{plt}\PY{o}{.}\PY{n}{imshow}\PY{p}{(}\PY{n}{threshed}\PY{p}{,} \PY{n}{cmap}\PY{o}{=}\PY{l+s+s1}{\PYZsq{}}\PY{l+s+s1}{gray}\PY{l+s+s1}{\PYZsq{}}\PY{p}{)}
        \PY{c+c1}{\PYZsh{}scipy.misc.imsave(\PYZsq{}../output/warped\PYZus{}threshed.jpg\PYZsq{}, threshed*255)}
\end{Verbatim}


\begin{Verbatim}[commandchars=\\\{\}]
{\color{outcolor}Out[{\color{outcolor}6}]:} <matplotlib.image.AxesImage at 0x1d282ac86a0>
\end{Verbatim}
            
    \begin{center}
    \adjustimage{max size={0.9\linewidth}{0.9\paperheight}}{output_10_1.png}
    \end{center}
    { \hspace*{\fill} \\}
    
    \subsection{Coordinate
Transformations}\label{coordinate-transformations}

Define the functions used to do coordinate transforms and apply them to
an image.

    \begin{Verbatim}[commandchars=\\\{\}]
{\color{incolor}In [{\color{incolor}7}]:} \PY{c+c1}{\PYZsh{} Define a function to convert from image coords to rover coords}
        \PY{k}{def} \PY{n+nf}{rover\PYZus{}coords}\PY{p}{(}\PY{n}{binary\PYZus{}img}\PY{p}{)}\PY{p}{:}
            \PY{c+c1}{\PYZsh{} Identify nonzero pixels}
            \PY{n}{ypos}\PY{p}{,} \PY{n}{xpos} \PY{o}{=} \PY{n}{binary\PYZus{}img}\PY{o}{.}\PY{n}{nonzero}\PY{p}{(}\PY{p}{)}
            \PY{c+c1}{\PYZsh{} Calculate pixel positions with reference to the rover position being at the }
            \PY{c+c1}{\PYZsh{} center bottom of the image.  }
            \PY{n}{x\PYZus{}pixel} \PY{o}{=} \PY{o}{\PYZhy{}}\PY{p}{(}\PY{n}{ypos} \PY{o}{\PYZhy{}} \PY{n}{binary\PYZus{}img}\PY{o}{.}\PY{n}{shape}\PY{p}{[}\PY{l+m+mi}{0}\PY{p}{]}\PY{p}{)}\PY{o}{.}\PY{n}{astype}\PY{p}{(}\PY{n}{np}\PY{o}{.}\PY{n}{float}\PY{p}{)}
            \PY{n}{y\PYZus{}pixel} \PY{o}{=} \PY{o}{\PYZhy{}}\PY{p}{(}\PY{n}{xpos} \PY{o}{\PYZhy{}} \PY{n}{binary\PYZus{}img}\PY{o}{.}\PY{n}{shape}\PY{p}{[}\PY{l+m+mi}{1}\PY{p}{]}\PY{o}{/}\PY{l+m+mi}{2} \PY{p}{)}\PY{o}{.}\PY{n}{astype}\PY{p}{(}\PY{n}{np}\PY{o}{.}\PY{n}{float}\PY{p}{)}
            \PY{k}{return} \PY{n}{x\PYZus{}pixel}\PY{p}{,} \PY{n}{y\PYZus{}pixel}
        
        \PY{c+c1}{\PYZsh{} Define a function to convert to radial coords in rover space}
        \PY{k}{def} \PY{n+nf}{to\PYZus{}polar\PYZus{}coords}\PY{p}{(}\PY{n}{x\PYZus{}pixel}\PY{p}{,} \PY{n}{y\PYZus{}pixel}\PY{p}{)}\PY{p}{:}
            \PY{c+c1}{\PYZsh{} Convert (x\PYZus{}pixel, y\PYZus{}pixel) to (distance, angle) }
            \PY{c+c1}{\PYZsh{} in polar coordinates in rover space}
            \PY{c+c1}{\PYZsh{} Calculate distance to each pixel}
            \PY{n}{dist} \PY{o}{=} \PY{n}{np}\PY{o}{.}\PY{n}{sqrt}\PY{p}{(}\PY{n}{x\PYZus{}pixel}\PY{o}{*}\PY{o}{*}\PY{l+m+mi}{2} \PY{o}{+} \PY{n}{y\PYZus{}pixel}\PY{o}{*}\PY{o}{*}\PY{l+m+mi}{2}\PY{p}{)}
            \PY{c+c1}{\PYZsh{} Calculate angle away from vertical for each pixel}
            \PY{n}{angles} \PY{o}{=} \PY{n}{np}\PY{o}{.}\PY{n}{arctan2}\PY{p}{(}\PY{n}{y\PYZus{}pixel}\PY{p}{,} \PY{n}{x\PYZus{}pixel}\PY{p}{)}
            \PY{k}{return} \PY{n}{dist}\PY{p}{,} \PY{n}{angles}
        
        \PY{c+c1}{\PYZsh{} Define a function to map rover space pixels to world space}
        \PY{k}{def} \PY{n+nf}{rotate\PYZus{}pix}\PY{p}{(}\PY{n}{xpix}\PY{p}{,} \PY{n}{ypix}\PY{p}{,} \PY{n}{yaw}\PY{p}{)}\PY{p}{:}
            \PY{c+c1}{\PYZsh{} Convert yaw to radians}
            \PY{n}{yaw\PYZus{}rad} \PY{o}{=} \PY{n}{yaw} \PY{o}{*} \PY{n}{np}\PY{o}{.}\PY{n}{pi} \PY{o}{/} \PY{l+m+mi}{180}
            \PY{n}{xpix\PYZus{}rotated} \PY{o}{=} \PY{p}{(}\PY{n}{xpix} \PY{o}{*} \PY{n}{np}\PY{o}{.}\PY{n}{cos}\PY{p}{(}\PY{n}{yaw\PYZus{}rad}\PY{p}{)}\PY{p}{)} \PY{o}{\PYZhy{}} \PY{p}{(}\PY{n}{ypix} \PY{o}{*} \PY{n}{np}\PY{o}{.}\PY{n}{sin}\PY{p}{(}\PY{n}{yaw\PYZus{}rad}\PY{p}{)}\PY{p}{)}
                                    
            \PY{n}{ypix\PYZus{}rotated} \PY{o}{=} \PY{p}{(}\PY{n}{xpix} \PY{o}{*} \PY{n}{np}\PY{o}{.}\PY{n}{sin}\PY{p}{(}\PY{n}{yaw\PYZus{}rad}\PY{p}{)}\PY{p}{)} \PY{o}{+} \PY{p}{(}\PY{n}{ypix} \PY{o}{*} \PY{n}{np}\PY{o}{.}\PY{n}{cos}\PY{p}{(}\PY{n}{yaw\PYZus{}rad}\PY{p}{)}\PY{p}{)}
            \PY{c+c1}{\PYZsh{} Return the result  }
            \PY{k}{return} \PY{n}{xpix\PYZus{}rotated}\PY{p}{,} \PY{n}{ypix\PYZus{}rotated}
        
        \PY{k}{def} \PY{n+nf}{translate\PYZus{}pix}\PY{p}{(}\PY{n}{xpix\PYZus{}rot}\PY{p}{,} \PY{n}{ypix\PYZus{}rot}\PY{p}{,} \PY{n}{xpos}\PY{p}{,} \PY{n}{ypos}\PY{p}{,} \PY{n}{scale}\PY{p}{)}\PY{p}{:} 
            \PY{c+c1}{\PYZsh{} Apply a scaling and a translation}
            \PY{n}{xpix\PYZus{}translated} \PY{o}{=} \PY{p}{(}\PY{n}{xpix\PYZus{}rot} \PY{o}{/} \PY{n}{scale}\PY{p}{)} \PY{o}{+} \PY{n}{xpos}
            \PY{n}{ypix\PYZus{}translated} \PY{o}{=} \PY{p}{(}\PY{n}{ypix\PYZus{}rot} \PY{o}{/} \PY{n}{scale}\PY{p}{)} \PY{o}{+} \PY{n}{ypos}
            \PY{c+c1}{\PYZsh{} Return the result  }
            \PY{k}{return} \PY{n}{xpix\PYZus{}translated}\PY{p}{,} \PY{n}{ypix\PYZus{}translated}
        
        
        \PY{c+c1}{\PYZsh{} Define a function to apply rotation and translation (and clipping)}
        \PY{c+c1}{\PYZsh{} Once you define the two functions above this function should work}
        \PY{k}{def} \PY{n+nf}{pix\PYZus{}to\PYZus{}world}\PY{p}{(}\PY{n}{xpix}\PY{p}{,} \PY{n}{ypix}\PY{p}{,} \PY{n}{xpos}\PY{p}{,} \PY{n}{ypos}\PY{p}{,} \PY{n}{yaw}\PY{p}{,} \PY{n}{world\PYZus{}size}\PY{p}{,} \PY{n}{scale}\PY{p}{)}\PY{p}{:}
            \PY{c+c1}{\PYZsh{} Apply rotation}
            \PY{n}{xpix\PYZus{}rot}\PY{p}{,} \PY{n}{ypix\PYZus{}rot} \PY{o}{=} \PY{n}{rotate\PYZus{}pix}\PY{p}{(}\PY{n}{xpix}\PY{p}{,} \PY{n}{ypix}\PY{p}{,} \PY{n}{yaw}\PY{p}{)}
            \PY{c+c1}{\PYZsh{} Apply translation}
            \PY{n}{xpix\PYZus{}tran}\PY{p}{,} \PY{n}{ypix\PYZus{}tran} \PY{o}{=} \PY{n}{translate\PYZus{}pix}\PY{p}{(}\PY{n}{xpix\PYZus{}rot}\PY{p}{,} \PY{n}{ypix\PYZus{}rot}\PY{p}{,} \PY{n}{xpos}\PY{p}{,} \PY{n}{ypos}\PY{p}{,} \PY{n}{scale}\PY{p}{)}
            \PY{c+c1}{\PYZsh{} Perform rotation, translation and clipping all at once}
            \PY{n}{x\PYZus{}pix\PYZus{}world} \PY{o}{=} \PY{n}{np}\PY{o}{.}\PY{n}{clip}\PY{p}{(}\PY{n}{np}\PY{o}{.}\PY{n}{int\PYZus{}}\PY{p}{(}\PY{n}{xpix\PYZus{}tran}\PY{p}{)}\PY{p}{,} \PY{l+m+mi}{0}\PY{p}{,} \PY{n}{world\PYZus{}size} \PY{o}{\PYZhy{}} \PY{l+m+mi}{1}\PY{p}{)}
            \PY{n}{y\PYZus{}pix\PYZus{}world} \PY{o}{=} \PY{n}{np}\PY{o}{.}\PY{n}{clip}\PY{p}{(}\PY{n}{np}\PY{o}{.}\PY{n}{int\PYZus{}}\PY{p}{(}\PY{n}{ypix\PYZus{}tran}\PY{p}{)}\PY{p}{,} \PY{l+m+mi}{0}\PY{p}{,} \PY{n}{world\PYZus{}size} \PY{o}{\PYZhy{}} \PY{l+m+mi}{1}\PY{p}{)}
            \PY{c+c1}{\PYZsh{} Return the result}
            \PY{k}{return} \PY{n}{x\PYZus{}pix\PYZus{}world}\PY{p}{,} \PY{n}{y\PYZus{}pix\PYZus{}world}
        
        \PY{c+c1}{\PYZsh{} Grab another random image}
        \PY{n}{idx} \PY{o}{=} \PY{n}{np}\PY{o}{.}\PY{n}{random}\PY{o}{.}\PY{n}{randint}\PY{p}{(}\PY{l+m+mi}{0}\PY{p}{,} \PY{n+nb}{len}\PY{p}{(}\PY{n}{img\PYZus{}list}\PY{p}{)}\PY{o}{\PYZhy{}}\PY{l+m+mi}{1}\PY{p}{)}
        \PY{n}{image} \PY{o}{=} \PY{n}{mpimg}\PY{o}{.}\PY{n}{imread}\PY{p}{(}\PY{n}{img\PYZus{}list}\PY{p}{[}\PY{n}{idx}\PY{p}{]}\PY{p}{)}
        \PY{n}{warped} \PY{o}{=} \PY{n}{perspect\PYZus{}transform}\PY{p}{(}\PY{n}{image}\PY{p}{,} \PY{n}{source}\PY{p}{,} \PY{n}{destination}\PY{p}{)}
        \PY{n}{threshed} \PY{o}{=} \PY{n}{color\PYZus{}thresh}\PY{p}{(}\PY{n}{warped}\PY{p}{)}
        
        \PY{c+c1}{\PYZsh{} Calculate pixel values in rover\PYZhy{}centric coords and distance/angle to all pixels}
        \PY{n}{xpix}\PY{p}{,} \PY{n}{ypix} \PY{o}{=} \PY{n}{rover\PYZus{}coords}\PY{p}{(}\PY{n}{threshed}\PY{p}{)}
        \PY{n}{dist}\PY{p}{,} \PY{n}{angles} \PY{o}{=} \PY{n}{to\PYZus{}polar\PYZus{}coords}\PY{p}{(}\PY{n}{xpix}\PY{p}{,} \PY{n}{ypix}\PY{p}{)}
        \PY{n}{mean\PYZus{}dir} \PY{o}{=} \PY{n}{np}\PY{o}{.}\PY{n}{mean}\PY{p}{(}\PY{n}{angles}\PY{p}{)}
        
        \PY{c+c1}{\PYZsh{} Do some plotting}
        \PY{n}{fig} \PY{o}{=} \PY{n}{plt}\PY{o}{.}\PY{n}{figure}\PY{p}{(}\PY{n}{figsize}\PY{o}{=}\PY{p}{(}\PY{l+m+mi}{12}\PY{p}{,}\PY{l+m+mi}{9}\PY{p}{)}\PY{p}{)}
        \PY{n}{plt}\PY{o}{.}\PY{n}{subplot}\PY{p}{(}\PY{l+m+mi}{221}\PY{p}{)}
        \PY{n}{plt}\PY{o}{.}\PY{n}{imshow}\PY{p}{(}\PY{n}{image}\PY{p}{)}
        \PY{n}{plt}\PY{o}{.}\PY{n}{subplot}\PY{p}{(}\PY{l+m+mi}{222}\PY{p}{)}
        \PY{n}{plt}\PY{o}{.}\PY{n}{imshow}\PY{p}{(}\PY{n}{warped}\PY{p}{)}
        \PY{n}{plt}\PY{o}{.}\PY{n}{subplot}\PY{p}{(}\PY{l+m+mi}{223}\PY{p}{)}
        \PY{n}{plt}\PY{o}{.}\PY{n}{imshow}\PY{p}{(}\PY{n}{threshed}\PY{p}{,} \PY{n}{cmap}\PY{o}{=}\PY{l+s+s1}{\PYZsq{}}\PY{l+s+s1}{gray}\PY{l+s+s1}{\PYZsq{}}\PY{p}{)}
        \PY{n}{plt}\PY{o}{.}\PY{n}{subplot}\PY{p}{(}\PY{l+m+mi}{224}\PY{p}{)}
        \PY{n}{plt}\PY{o}{.}\PY{n}{plot}\PY{p}{(}\PY{n}{xpix}\PY{p}{,} \PY{n}{ypix}\PY{p}{,} \PY{l+s+s1}{\PYZsq{}}\PY{l+s+s1}{.}\PY{l+s+s1}{\PYZsq{}}\PY{p}{)}
        \PY{n}{plt}\PY{o}{.}\PY{n}{ylim}\PY{p}{(}\PY{o}{\PYZhy{}}\PY{l+m+mi}{160}\PY{p}{,} \PY{l+m+mi}{160}\PY{p}{)}
        \PY{n}{plt}\PY{o}{.}\PY{n}{xlim}\PY{p}{(}\PY{l+m+mi}{0}\PY{p}{,} \PY{l+m+mi}{160}\PY{p}{)}
        \PY{n}{arrow\PYZus{}length} \PY{o}{=} \PY{l+m+mi}{100}
        \PY{n}{x\PYZus{}arrow} \PY{o}{=} \PY{n}{arrow\PYZus{}length} \PY{o}{*} \PY{n}{np}\PY{o}{.}\PY{n}{cos}\PY{p}{(}\PY{n}{mean\PYZus{}dir}\PY{p}{)}
        \PY{n}{y\PYZus{}arrow} \PY{o}{=} \PY{n}{arrow\PYZus{}length} \PY{o}{*} \PY{n}{np}\PY{o}{.}\PY{n}{sin}\PY{p}{(}\PY{n}{mean\PYZus{}dir}\PY{p}{)}
        \PY{n}{plt}\PY{o}{.}\PY{n}{arrow}\PY{p}{(}\PY{l+m+mi}{0}\PY{p}{,} \PY{l+m+mi}{0}\PY{p}{,} \PY{n}{x\PYZus{}arrow}\PY{p}{,} \PY{n}{y\PYZus{}arrow}\PY{p}{,} \PY{n}{color}\PY{o}{=}\PY{l+s+s1}{\PYZsq{}}\PY{l+s+s1}{red}\PY{l+s+s1}{\PYZsq{}}\PY{p}{,} \PY{n}{zorder}\PY{o}{=}\PY{l+m+mi}{2}\PY{p}{,} \PY{n}{head\PYZus{}width}\PY{o}{=}\PY{l+m+mi}{10}\PY{p}{,} \PY{n}{width}\PY{o}{=}\PY{l+m+mi}{2}\PY{p}{)}
\end{Verbatim}


\begin{Verbatim}[commandchars=\\\{\}]
{\color{outcolor}Out[{\color{outcolor}7}]:} <matplotlib.patches.FancyArrow at 0x1d282b7c828>
\end{Verbatim}
            
    \begin{center}
    \adjustimage{max size={0.9\linewidth}{0.9\paperheight}}{output_12_1.png}
    \end{center}
    { \hspace*{\fill} \\}
    
    \subsection{Read in saved data and ground truth map of the
world}\label{read-in-saved-data-and-ground-truth-map-of-the-world}

The next cell is all setup to read your saved data into a
\texttt{pandas} dataframe. Here you'll also read in a "ground truth" map
of the world, where white pixels (pixel value = 1) represent navigable
terrain.

After that, we'll define a class to store telemetry data and pathnames
to images. When you instantiate this class
(\texttt{data\ =\ Databucket()}) you'll have a global variable called
\texttt{data} that you can refer to for telemetry and map data within
the \texttt{process\_image()} function in the following cell.

    \begin{Verbatim}[commandchars=\\\{\}]
{\color{incolor}In [{\color{incolor}8}]:} \PY{c+c1}{\PYZsh{} Import pandas and read in csv file as a dataframe}
        \PY{k+kn}{import} \PY{n+nn}{pandas} \PY{k}{as} \PY{n+nn}{pd}
        \PY{c+c1}{\PYZsh{} Change the path below to your data directory}
        \PY{c+c1}{\PYZsh{} If you are in a locale (e.g., Europe) that uses \PYZsq{},\PYZsq{} as the decimal separator}
        \PY{c+c1}{\PYZsh{} change the \PYZsq{}.\PYZsq{} to \PYZsq{},\PYZsq{}}
        \PY{n}{df} \PY{o}{=} \PY{n}{pd}\PY{o}{.}\PY{n}{read\PYZus{}csv}\PY{p}{(}\PY{l+s+s1}{\PYZsq{}}\PY{l+s+s1}{../test\PYZus{}dataset/robot\PYZus{}log.csv}\PY{l+s+s1}{\PYZsq{}}\PY{p}{,} \PY{n}{delimiter}\PY{o}{=}\PY{l+s+s1}{\PYZsq{}}\PY{l+s+s1}{;}\PY{l+s+s1}{\PYZsq{}}\PY{p}{,} \PY{n}{decimal}\PY{o}{=}\PY{l+s+s1}{\PYZsq{}}\PY{l+s+s1}{.}\PY{l+s+s1}{\PYZsq{}}\PY{p}{)}
        \PY{n}{csv\PYZus{}img\PYZus{}list} \PY{o}{=} \PY{n}{df}\PY{p}{[}\PY{l+s+s2}{\PYZdq{}}\PY{l+s+s2}{Path}\PY{l+s+s2}{\PYZdq{}}\PY{p}{]}\PY{o}{.}\PY{n}{tolist}\PY{p}{(}\PY{p}{)} \PY{c+c1}{\PYZsh{} Create list of image pathnames}
        \PY{c+c1}{\PYZsh{} Read in ground truth map and create a 3\PYZhy{}channel image with it}
        \PY{n}{ground\PYZus{}truth} \PY{o}{=} \PY{n}{mpimg}\PY{o}{.}\PY{n}{imread}\PY{p}{(}\PY{l+s+s1}{\PYZsq{}}\PY{l+s+s1}{../calibration\PYZus{}images/map\PYZus{}bw.png}\PY{l+s+s1}{\PYZsq{}}\PY{p}{)}
        \PY{n}{ground\PYZus{}truth\PYZus{}3d} \PY{o}{=} \PY{n}{np}\PY{o}{.}\PY{n}{dstack}\PY{p}{(}\PY{p}{(}\PY{n}{ground\PYZus{}truth}\PY{o}{*}\PY{l+m+mi}{0}\PY{p}{,} \PY{n}{ground\PYZus{}truth}\PY{o}{*}\PY{l+m+mi}{255}\PY{p}{,} \PY{n}{ground\PYZus{}truth}\PY{o}{*}\PY{l+m+mi}{0}\PY{p}{)}\PY{p}{)}\PY{o}{.}\PY{n}{astype}\PY{p}{(}\PY{n}{np}\PY{o}{.}\PY{n}{float}\PY{p}{)}
        
        \PY{c+c1}{\PYZsh{} Creating a class to be the data container}
        \PY{c+c1}{\PYZsh{} Will read in saved data from csv file and populate this object}
        \PY{c+c1}{\PYZsh{} Worldmap is instantiated as 200 x 200 grids corresponding }
        \PY{c+c1}{\PYZsh{} to a 200m x 200m space (same size as the ground truth map: 200 x 200 pixels)}
        \PY{c+c1}{\PYZsh{} This encompasses the full range of output position values in x and y from the sim}
        \PY{k}{class} \PY{n+nc}{Databucket}\PY{p}{(}\PY{p}{)}\PY{p}{:}
            \PY{k}{def} \PY{n+nf}{\PYZus{}\PYZus{}init\PYZus{}\PYZus{}}\PY{p}{(}\PY{n+nb+bp}{self}\PY{p}{)}\PY{p}{:}
                \PY{n+nb+bp}{self}\PY{o}{.}\PY{n}{images} \PY{o}{=} \PY{n}{csv\PYZus{}img\PYZus{}list}  
                \PY{n+nb+bp}{self}\PY{o}{.}\PY{n}{xpos} \PY{o}{=} \PY{n}{df}\PY{p}{[}\PY{l+s+s2}{\PYZdq{}}\PY{l+s+s2}{X\PYZus{}Position}\PY{l+s+s2}{\PYZdq{}}\PY{p}{]}\PY{o}{.}\PY{n}{values}
                \PY{n+nb+bp}{self}\PY{o}{.}\PY{n}{ypos} \PY{o}{=} \PY{n}{df}\PY{p}{[}\PY{l+s+s2}{\PYZdq{}}\PY{l+s+s2}{Y\PYZus{}Position}\PY{l+s+s2}{\PYZdq{}}\PY{p}{]}\PY{o}{.}\PY{n}{values}
                \PY{n+nb+bp}{self}\PY{o}{.}\PY{n}{yaw} \PY{o}{=} \PY{n}{df}\PY{p}{[}\PY{l+s+s2}{\PYZdq{}}\PY{l+s+s2}{Yaw}\PY{l+s+s2}{\PYZdq{}}\PY{p}{]}\PY{o}{.}\PY{n}{values}
                \PY{n+nb+bp}{self}\PY{o}{.}\PY{n}{count} \PY{o}{=} \PY{l+m+mi}{0} \PY{c+c1}{\PYZsh{} This will be a running index}
                \PY{n+nb+bp}{self}\PY{o}{.}\PY{n}{worldmap} \PY{o}{=} \PY{n}{np}\PY{o}{.}\PY{n}{zeros}\PY{p}{(}\PY{p}{(}\PY{l+m+mi}{200}\PY{p}{,} \PY{l+m+mi}{200}\PY{p}{,} \PY{l+m+mi}{3}\PY{p}{)}\PY{p}{)}\PY{o}{.}\PY{n}{astype}\PY{p}{(}\PY{n}{np}\PY{o}{.}\PY{n}{float}\PY{p}{)}
                \PY{n+nb+bp}{self}\PY{o}{.}\PY{n}{ground\PYZus{}truth} \PY{o}{=} \PY{n}{ground\PYZus{}truth\PYZus{}3d} \PY{c+c1}{\PYZsh{} Ground truth worldmap}
        
        \PY{c+c1}{\PYZsh{} Instantiate a Databucket().. this will be a global variable/object}
        \PY{c+c1}{\PYZsh{} that you can refer to in the process\PYZus{}image() function below}
        \PY{n}{data} \PY{o}{=} \PY{n}{Databucket}\PY{p}{(}\PY{p}{)}
\end{Verbatim}


    \subsection{Write a function to process stored
images}\label{write-a-function-to-process-stored-images}

Modify the \texttt{process\_image()} function below by adding in the
perception step processes (functions defined above) to perform image
analysis and mapping. The following cell is all set up to use this
\texttt{process\_image()} function in conjunction with the
\texttt{moviepy} video processing package to create a video from the
images you saved taking data in the simulator.

In short, you will be passing individual images into
\texttt{process\_image()} and building up an image called
\texttt{output\_image} that will be stored as one frame of video. You
can make a mosaic of the various steps of your analysis process and add
text as you like (example provided below).

To start with, you can simply run the next three cells to see what
happens, but then go ahead and modify them such that the output video
demonstrates your mapping process. Feel free to get creative!

    \begin{Verbatim}[commandchars=\\\{\}]
{\color{incolor}In [{\color{incolor}9}]:} \PY{c+c1}{\PYZsh{} Define a function to pass stored images to}
        \PY{c+c1}{\PYZsh{} reading rover position and yaw angle from csv file}
        \PY{c+c1}{\PYZsh{} This function will be used by moviepy to create an output video}
        \PY{k}{def} \PY{n+nf}{process\PYZus{}image}\PY{p}{(}\PY{n}{img}\PY{p}{)}\PY{p}{:}
            \PY{c+c1}{\PYZsh{} Example of how to use the Databucket() object defined above}
            \PY{c+c1}{\PYZsh{} to print the current x, y and yaw values }
            \PY{c+c1}{\PYZsh{} print(data.xpos[data.count], data.ypos[data.count], data.yaw[data.count])}
        
            \PY{c+c1}{\PYZsh{} TODO: }
            \PY{c+c1}{\PYZsh{} 1) Define source and destination points for perspective transform}
            \PY{c+c1}{\PYZsh{} 2) Apply perspective transform}
            \PY{c+c1}{\PYZsh{} 3) Apply color threshold to identify navigable terrain/obstacles/rock samples}
            \PY{c+c1}{\PYZsh{} 4) Convert thresholded image pixel values to rover\PYZhy{}centric coords}
            \PY{c+c1}{\PYZsh{} 5) Convert rover\PYZhy{}centric pixel values to world coords}
            \PY{c+c1}{\PYZsh{} 6) Update worldmap (to be displayed on right side of screen)}
                \PY{c+c1}{\PYZsh{} Example: data.worldmap[obstacle\PYZus{}y\PYZus{}world, obstacle\PYZus{}x\PYZus{}world, 0] += 1}
                \PY{c+c1}{\PYZsh{}          data.worldmap[rock\PYZus{}y\PYZus{}world, rock\PYZus{}x\PYZus{}world, 1] += 1}
                \PY{c+c1}{\PYZsh{}          data.worldmap[navigable\PYZus{}y\PYZus{}world, navigable\PYZus{}x\PYZus{}world, 2] += 1}
        
            \PY{c+c1}{\PYZsh{} 7) Make a mosaic image, below is some example code}
                \PY{c+c1}{\PYZsh{} First create a blank image (can be whatever shape you like)}
            \PY{n}{output\PYZus{}image} \PY{o}{=} \PY{n}{np}\PY{o}{.}\PY{n}{zeros}\PY{p}{(}\PY{p}{(}\PY{n}{img}\PY{o}{.}\PY{n}{shape}\PY{p}{[}\PY{l+m+mi}{0}\PY{p}{]} \PY{o}{+} \PY{n}{data}\PY{o}{.}\PY{n}{worldmap}\PY{o}{.}\PY{n}{shape}\PY{p}{[}\PY{l+m+mi}{0}\PY{p}{]}\PY{p}{,} \PY{n}{img}\PY{o}{.}\PY{n}{shape}\PY{p}{[}\PY{l+m+mi}{1}\PY{p}{]}\PY{o}{*}\PY{l+m+mi}{2}\PY{p}{,} \PY{l+m+mi}{3}\PY{p}{)}\PY{p}{)}
                \PY{c+c1}{\PYZsh{} Next you can populate regions of the image with various output}
                \PY{c+c1}{\PYZsh{} Here I\PYZsq{}m putting the original image in the upper left hand corner}
            \PY{n}{output\PYZus{}image}\PY{p}{[}\PY{l+m+mi}{0}\PY{p}{:}\PY{n}{img}\PY{o}{.}\PY{n}{shape}\PY{p}{[}\PY{l+m+mi}{0}\PY{p}{]}\PY{p}{,} \PY{l+m+mi}{0}\PY{p}{:}\PY{n}{img}\PY{o}{.}\PY{n}{shape}\PY{p}{[}\PY{l+m+mi}{1}\PY{p}{]}\PY{p}{]} \PY{o}{=} \PY{n}{img}
        
                \PY{c+c1}{\PYZsh{} Let\PYZsq{}s create more images to add to the mosaic, first a warped image}
            \PY{n}{warped} \PY{o}{=} \PY{n}{perspect\PYZus{}transform}\PY{p}{(}\PY{n}{img}\PY{p}{,} \PY{n}{source}\PY{p}{,} \PY{n}{destination}\PY{p}{)}
                \PY{c+c1}{\PYZsh{} Add the warped image in the upper right hand corner}
            \PY{n}{output\PYZus{}image}\PY{p}{[}\PY{l+m+mi}{0}\PY{p}{:}\PY{n}{img}\PY{o}{.}\PY{n}{shape}\PY{p}{[}\PY{l+m+mi}{0}\PY{p}{]}\PY{p}{,} \PY{n}{img}\PY{o}{.}\PY{n}{shape}\PY{p}{[}\PY{l+m+mi}{1}\PY{p}{]}\PY{p}{:}\PY{p}{]} \PY{o}{=} \PY{n}{warped}
        
                \PY{c+c1}{\PYZsh{} Overlay worldmap with ground truth map}
            \PY{n}{map\PYZus{}add} \PY{o}{=} \PY{n}{cv2}\PY{o}{.}\PY{n}{addWeighted}\PY{p}{(}\PY{n}{data}\PY{o}{.}\PY{n}{worldmap}\PY{p}{,} \PY{l+m+mi}{1}\PY{p}{,} \PY{n}{data}\PY{o}{.}\PY{n}{ground\PYZus{}truth}\PY{p}{,} \PY{l+m+mf}{0.5}\PY{p}{,} \PY{l+m+mi}{0}\PY{p}{)}
                \PY{c+c1}{\PYZsh{} Flip map overlay so y\PYZhy{}axis points upward and add to output\PYZus{}image }
            \PY{n}{output\PYZus{}image}\PY{p}{[}\PY{n}{img}\PY{o}{.}\PY{n}{shape}\PY{p}{[}\PY{l+m+mi}{0}\PY{p}{]}\PY{p}{:}\PY{p}{,} \PY{l+m+mi}{0}\PY{p}{:}\PY{n}{data}\PY{o}{.}\PY{n}{worldmap}\PY{o}{.}\PY{n}{shape}\PY{p}{[}\PY{l+m+mi}{1}\PY{p}{]}\PY{p}{]} \PY{o}{=} \PY{n}{np}\PY{o}{.}\PY{n}{flipud}\PY{p}{(}\PY{n}{map\PYZus{}add}\PY{p}{)}
        
        
                \PY{c+c1}{\PYZsh{} Then putting some text over the image}
            \PY{n}{cv2}\PY{o}{.}\PY{n}{putText}\PY{p}{(}\PY{n}{output\PYZus{}image}\PY{p}{,}\PY{l+s+s2}{\PYZdq{}}\PY{l+s+s2}{Populate this image with your analyses to make a video!}\PY{l+s+s2}{\PYZdq{}}\PY{p}{,} \PY{p}{(}\PY{l+m+mi}{20}\PY{p}{,} \PY{l+m+mi}{20}\PY{p}{)}\PY{p}{,} 
                        \PY{n}{cv2}\PY{o}{.}\PY{n}{FONT\PYZus{}HERSHEY\PYZus{}COMPLEX}\PY{p}{,} \PY{l+m+mf}{0.4}\PY{p}{,} \PY{p}{(}\PY{l+m+mi}{255}\PY{p}{,} \PY{l+m+mi}{255}\PY{p}{,} \PY{l+m+mi}{255}\PY{p}{)}\PY{p}{,} \PY{l+m+mi}{1}\PY{p}{)}
            \PY{k}{if} \PY{n}{data}\PY{o}{.}\PY{n}{count} \PY{o}{\PYZlt{}} \PY{n+nb}{len}\PY{p}{(}\PY{n}{data}\PY{o}{.}\PY{n}{images}\PY{p}{)} \PY{o}{\PYZhy{}} \PY{l+m+mi}{1}\PY{p}{:}
                \PY{n}{data}\PY{o}{.}\PY{n}{count} \PY{o}{+}\PY{o}{=} \PY{l+m+mi}{1} \PY{c+c1}{\PYZsh{} Keep track of the index in the Databucket()}
            
            \PY{k}{return} \PY{n}{output\PYZus{}image}
\end{Verbatim}


    \subsection{Make a video from processed image
data}\label{make-a-video-from-processed-image-data}

Use the \href{https://zulko.github.io/moviepy/}{moviepy} library to
process images and create a video.

    \begin{Verbatim}[commandchars=\\\{\}]
{\color{incolor}In [{\color{incolor}11}]:} \PY{c+c1}{\PYZsh{} Import everything needed to edit/save/watch video clips}
         \PY{k+kn}{from} \PY{n+nn}{moviepy}\PY{n+nn}{.}\PY{n+nn}{editor} \PY{k}{import} \PY{n}{VideoFileClip}
         \PY{k+kn}{from} \PY{n+nn}{moviepy}\PY{n+nn}{.}\PY{n+nn}{editor} \PY{k}{import} \PY{n}{ImageSequenceClip}
         
         
         \PY{c+c1}{\PYZsh{} Define pathname to save the output video}
         \PY{n}{output} \PY{o}{=} \PY{l+s+s1}{\PYZsq{}}\PY{l+s+s1}{../output/test\PYZus{}mapping.mp4}\PY{l+s+s1}{\PYZsq{}}
         \PY{n}{data} \PY{o}{=} \PY{n}{Databucket}\PY{p}{(}\PY{p}{)} \PY{c+c1}{\PYZsh{} Re\PYZhy{}initialize data in case you\PYZsq{}re running this cell multiple times}
         \PY{n}{clip} \PY{o}{=} \PY{n}{ImageSequenceClip}\PY{p}{(}\PY{n}{data}\PY{o}{.}\PY{n}{images}\PY{p}{,} \PY{n}{fps}\PY{o}{=}\PY{l+m+mi}{60}\PY{p}{)} \PY{c+c1}{\PYZsh{} Note: output video will be sped up because }
                                                   \PY{c+c1}{\PYZsh{} recording rate in simulator is fps=25}
         \PY{n}{new\PYZus{}clip} \PY{o}{=} \PY{n}{clip}\PY{o}{.}\PY{n}{fl\PYZus{}image}\PY{p}{(}\PY{n}{process\PYZus{}image}\PY{p}{)} \PY{c+c1}{\PYZsh{}NOTE: this function expects color images!!}
         \PY{o}{\PYZpc{}}\PY{k}{time} new\PYZus{}clip.write\PYZus{}videofile(output, audio=False)
\end{Verbatim}


    \begin{Verbatim}[commandchars=\\\{\}]
[MoviePy] >>>> Building video ../output/test\_mapping.mp4
[MoviePy] Writing video ../output/test\_mapping.mp4

    \end{Verbatim}

    \begin{Verbatim}[commandchars=\\\{\}]
100\%|███████████████████████████████████████████████████████████████████████████████| 283/283 [00:02<00:00, 126.34it/s]

    \end{Verbatim}

    \begin{Verbatim}[commandchars=\\\{\}]
[MoviePy] Done.
[MoviePy] >>>> Video ready: ../output/test\_mapping.mp4 

Wall time: 2.46 s

    \end{Verbatim}

    \subsubsection{This next cell should function as an inline video
player}\label{this-next-cell-should-function-as-an-inline-video-player}

If this fails to render the video, try running the following cell
(alternative video rendering method). You can also simply have a look at
the saved mp4 in your \texttt{/output} folder

    \begin{Verbatim}[commandchars=\\\{\}]
{\color{incolor}In [{\color{incolor}12}]:} \PY{k+kn}{from} \PY{n+nn}{IPython}\PY{n+nn}{.}\PY{n+nn}{display} \PY{k}{import} \PY{n}{HTML}
         \PY{n}{HTML}\PY{p}{(}\PY{l+s+s2}{\PYZdq{}\PYZdq{}\PYZdq{}}
         \PY{l+s+s2}{\PYZlt{}video width=}\PY{l+s+s2}{\PYZdq{}}\PY{l+s+s2}{960}\PY{l+s+s2}{\PYZdq{}}\PY{l+s+s2}{ height=}\PY{l+s+s2}{\PYZdq{}}\PY{l+s+s2}{540}\PY{l+s+s2}{\PYZdq{}}\PY{l+s+s2}{ controls\PYZgt{}}
         \PY{l+s+s2}{  \PYZlt{}source src=}\PY{l+s+s2}{\PYZdq{}}\PY{l+s+si}{\PYZob{}0\PYZcb{}}\PY{l+s+s2}{\PYZdq{}}\PY{l+s+s2}{\PYZgt{}}
         \PY{l+s+s2}{\PYZlt{}/video\PYZgt{}}
         \PY{l+s+s2}{\PYZdq{}\PYZdq{}\PYZdq{}}\PY{o}{.}\PY{n}{format}\PY{p}{(}\PY{n}{output}\PY{p}{)}\PY{p}{)}
\end{Verbatim}


\begin{Verbatim}[commandchars=\\\{\}]
{\color{outcolor}Out[{\color{outcolor}12}]:} <IPython.core.display.HTML object>
\end{Verbatim}
            
    \subsubsection{Below is an alternative way to create a video in case the
above cell did not
work.}\label{below-is-an-alternative-way-to-create-a-video-in-case-the-above-cell-did-not-work.}

    \begin{Verbatim}[commandchars=\\\{\}]
{\color{incolor}In [{\color{incolor} }]:} \PY{k+kn}{import} \PY{n+nn}{io}
        \PY{k+kn}{import} \PY{n+nn}{base64}
        \PY{n}{video} \PY{o}{=} \PY{n}{io}\PY{o}{.}\PY{n}{open}\PY{p}{(}\PY{n}{output}\PY{p}{,} \PY{l+s+s1}{\PYZsq{}}\PY{l+s+s1}{r+b}\PY{l+s+s1}{\PYZsq{}}\PY{p}{)}\PY{o}{.}\PY{n}{read}\PY{p}{(}\PY{p}{)}
        \PY{n}{encoded\PYZus{}video} \PY{o}{=} \PY{n}{base64}\PY{o}{.}\PY{n}{b64encode}\PY{p}{(}\PY{n}{video}\PY{p}{)}
        \PY{n}{HTML}\PY{p}{(}\PY{n}{data}\PY{o}{=}\PY{l+s+s1}{\PYZsq{}\PYZsq{}\PYZsq{}}\PY{l+s+s1}{\PYZlt{}video alt=}\PY{l+s+s1}{\PYZdq{}}\PY{l+s+s1}{test}\PY{l+s+s1}{\PYZdq{}}\PY{l+s+s1}{ controls\PYZgt{}}
        \PY{l+s+s1}{                \PYZlt{}source src=}\PY{l+s+s1}{\PYZdq{}}\PY{l+s+s1}{data:video/mp4;base64,}\PY{l+s+si}{\PYZob{}0\PYZcb{}}\PY{l+s+s1}{\PYZdq{}}\PY{l+s+s1}{ type=}\PY{l+s+s1}{\PYZdq{}}\PY{l+s+s1}{video/mp4}\PY{l+s+s1}{\PYZdq{}}\PY{l+s+s1}{ /\PYZgt{}}
        \PY{l+s+s1}{             \PYZlt{}/video\PYZgt{}}\PY{l+s+s1}{\PYZsq{}\PYZsq{}\PYZsq{}}\PY{o}{.}\PY{n}{format}\PY{p}{(}\PY{n}{encoded\PYZus{}video}\PY{o}{.}\PY{n}{decode}\PY{p}{(}\PY{l+s+s1}{\PYZsq{}}\PY{l+s+s1}{ascii}\PY{l+s+s1}{\PYZsq{}}\PY{p}{)}\PY{p}{)}\PY{p}{)}
\end{Verbatim}



    % Add a bibliography block to the postdoc
    
    
    
    \end{document}
